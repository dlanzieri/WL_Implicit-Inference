
\documentclass{aa}  

%
\usepackage{graphicx}
\usepackage[table]{xcolor}
%%%%%%%%%%%%%%%%%%%%%%%%%%%%%%%%%%%%%%%%
\usepackage{txfonts}
\usepackage{bm}
\usepackage{multirow, makecell}
\usepackage{booktabs}
%%%%%%%%%%%%%%%%%%%%%%%%%%%%%%%%%%%%%%%%

\usepackage[colorlinks=true,citecolor=blue,linkcolor=blue,urlcolor=blue]{hyperref}
\begin{document} 

\title{Unveiling the Optimal Approach for Neural Network-Based Cosmological Parameter Inference through Likelihood Implicit Methods}





\author{Denise Lanzieri \inst{1}
\and
Justine Zeghal \inst{2}
\and
Alexandre Boucaud \inst{2}
\and
Fran\c{c}ois Lanusse \inst{3}
\and
Jean-Luc Starck \inst{3}
}
\institute{Université Paris Cité, Université Paris-Saclay, CEA, CNRS, AIM, F-91191, Gif-sur-Yvette, France
\and
Université Paris Cité, CNRS, Astroparticule et Cosmologie, F-75013 Paris, France
\and
Université Paris-Saclay, Université Paris Cité, CEA, CNRS, AIM, 91191, Gif-sur-Yvette, France
}
\titlerunning{}
\date{Received xxx; accepted xxx}


 
  \abstract
  % context heading (optional)
  % {} leave it empty if necessary  
   {blabla}
  % aims heading (mandatory)
    {blabla}
  % methods heading (mandatory)
    {blabla}
  % results heading (mandatory)
   {blabla}
  % conclusions heading (optional), leave it empty if necessary 
   {}

   \keywords{giant planet formation --
                $\kappa$-mechanism --
                stability of gas spheres
               }

   \maketitle
%
%-------------------------------------------------------------------

\section{Introduction}
\section{Theory Background}
\subsection{Weak lensing formalism}
\subsection{Lognormal Modeling}
For several cosmological purposes, the non-Gaussian field can be modeled as a Lognormal field \citep{coles1991lognormal,bohm2017bayesian}.
This model has the advantage of offering the rapid generation of the matter or the convergence field while allowing the extraction of information beyond the Gaussian one. 
Although several studies demonstrated that this model fails in describing the 3D field, it properly describes the 2D convergence field.



Assume a multivariate Gaussian distribution G described by the vector elements of mean $\mu_i$ and covariance matrix $\xi_g^{ij}$, we can define the density contrast $\delta(\bm{x})$ as a shifted lognormal random field :
% The distribution of the convergence $\kappa_i$ in the $i-$th redshift bin can be modeled as:
\begin{equation}
    \delta(\bm{x})=\lambda[e^{g(\bm{x})}-1]. 
\end{equation}\label{Eq:log_norm_kappa}
\begin{equation}\label{Eq:log_norm_kappa}
    \kappa_{i}=e^{y_i}-\lambda_i. 
\end{equation}
where $\lambda_i$ is a free parameter defining the shift of the lognormal distribution.  The density contrast $\delta(\bm{x})$ is fully defined by the shift parameter $\lambda$, the mean of the associated Gaussian field $\mu_i$, and its variance $\sigma_g^2\equiv \xi_g$. That fact that we require $<\delta>=0$ implies that the Gaussian field satisfies the following relation:
\begin{equation}
    \mu_g=-\frac{\sigma_g^2}{2}.
\end{equation}



The parameter $\lambda_i$, is also known as \textit{minimum convergence
parameter}, since defines the lowest values for all possible values $\kappa$ being the lognormal defined of for positive values. We can see from \autoref{Eq:log_norm_kappa} that th ereandom variables $\kappa_i$ is fully defined by the shift parameter $\lambda_i$, the mean of the associated gaussian variables $\mu_i$, and its variance $\sigma_i^2\equiv \xi_g^{ii}$. 

 The modelling for the shift parameter can be approached in different ways.  determine
For example, $\lambda$ can be determined following the approach of \citet{xavier2016improving} , i.e.  matching moments of the distribution, or following the approach of \citet{hilbert2011cosmic}, i.e. fitting it as a free parameter. In general, it is a parameter dependent from the redshift, the cosmology, and the scale of the field at which the smoothing is applied.
In this work, we choose the derived $\lambda$ using Cosmomentum.
The lognormal $\xi^{ij}_{ln}$ and the associated Gaussian covariances $ \xi^{ij}_g$ are related via :
\begin{align}
    \xi^{ij}_{ln}(\theta) & \equiv \lambda_i \lambda_j (e^{ \xi^{ij}_g(\theta)}-1) \nonumber \\ 
    \xi^{ij}_g(\theta)&=\log{\left[ \frac{\xi^{ij}_{ln}(\theta)}{\lambda_i \lambda_j}+1\right ]}. \label{Eq:log_norm_corr}
\end{align}
%Note for Denise: The lambda_i and the \lambda_j in Eq:log_norm_corr are the same parameters defined in Eq:log_norm_kappa only for Gaussian distribution with mean zero. Otherwise, these parameters should be defined as alpha_i and alpha_j where alpha_i= <y>+lambda_i. See Xavier et al. 2015 for a detailed description. 

One of the advantages of modeling the convergence field through a lognormal model is the fact that it enables the constraining of the convergence field in different redshift bins simultaneously while taking into account the correlation between the bins. Indeed, in the Fourier space, the power spectrum of $y$ is defined as:
\begin{equation}\label{Eq:log_norm_cls}
    C^{ij}_y(\ell)=2\pi \int_0^{\pi} d\theta \sin{\theta}P_{\ell}(\cos{\theta})\xi^{ij}_{y}(\theta)
\end{equation}
with $P_{\ell}$ Legendre polynomial with order $\ell$. 

Having defined the survey to reproduce in terms of galaxy number density, redshifts, and shape noise, we can sample the Lognormal random map $\kappa$.  using \autoref{Eq:log_norm_kappa} and \autoref{Eq:log_norm_corr}.  First, we need to compute the theoretical auto $C_{\ell}^{i}$ and cross-correlated $C_{\ell}^{ij}$ angular power spectrum for each tomographic bin. These theoretical predictions are computed using the public library  \href{https://github.com/DifferentiableUniverseInitiative/jax_cosmo}{\texttt{jax-mosmo}}. Then the 1-D $C_{\ell}$ are projected into 2-D grids of the same size as the desired final convergence maps and the Gaussian covariances $\xi^{ij}_y(\theta)$ are computed through \autoref{Eq:log_norm_corr}. To take into account the cross-correlation between different redshift bins, we first stack together the vector $\xi^{ij}_y(\theta)$ to build a covariance matrix having the auto-correlation on the main diagonal and the cross-correlation off-diagonal. Then we perform an eigenvalues decomposition. This decomposition enables us to decorrelate the maps

\begin{equation}
    \bm{\Sigma}= 
    \begin{pmatrix}
    C_{\ell}^{11} & C_{\ell}^{12} & \cdots & C_{\ell}^{1n} \\
    C_{\ell}^{21} & C_{\ell}^{22} & \cdots & C_{\ell}^{2n} \\
    \vdots  & \vdots  & \ddots & \vdots  \\
    C_{\ell}^{n1} & C_{\ell}^{n2} & \cdots & C_{\ell}^{nn} 
    \end{pmatrix}
\end{equation}
%--------------------------------------------------------------------
\section{Simulations}
\subsection{Data generation}
Our analysis is based on a standard flat $\Lambda$CDM cosmological model, with the following parameters: the baryonic density fraction $\Omega_b$, the total matter density fraction $\Omega_m$, the Hubble parameter $h_0$, the spectral index $n_s$, the amplitude of the primordial power spectrum $\sigma_8$ and the dark energy parameter $w_0$. The priors used in the simulations and in inference process are listed in \autoref{tab:prior}, according to \citet{zhang2022transitioning}.
We compute the theoretical power and cross spectra using the public library \href{https://github.com/DifferentiableUniverseInitiative/jax_cosmo}{\texttt{jax-cosmo}} \citep{Campagne_2023}. To calculate the 
lognormal shift parameter, we adopt the \texttt{Cosmomentum} code \citep{friedrich2018density, friedrich2020primordial}, which utilizes perturbation theory to compute the cosmology-dependent shift parameters. Therefore, the calculation of the shift parameters is based on a window function, while our pixels are rectangular. Following, we compute the shift parameters at a characteristic scale, $R=\Delta L/\pi$  with $\Delta L$, the resolution of the pixels.  It is important to note that, as pointed out by \citet{boruah2022map}, this perturbation theory-based approach may not provide accurate results at small scales. However, since the main objective of this paper is to compare different compression strategies, we are not concerned about the potential implications of this approximation. In future applications of the presented inference pipeline, the simulation procedure will be based on N-body simulations.


Simulating the log-normal field for one single convergence map is a straightforward process. 
However, for a tomographic analysis, as described in the previous section, it is necessary to consider the cross-correlation between different tomographic bins. To account for this correlation, we employ the sampling strategy described in Section X. \\
Each map is reproduced on a regular grid with dimensions of $256^2$ pixels and covers an area of $10\times 10$ deg$^2$.

\begin{table}
	\centering
	\caption{ Prior and fiducial values used for the analyses. 
 The symbol $\mathcal{N}_T$ represents a Truncated Normal distribution. The lower bound of the support for the $\Omega_c$ distribution is set to zero, while the lower and upper bounds for the $w_0$ distribution are set to -2.0 and -0.3, respectively.}
	\begin{tabular}{lcc} 
		\hline \hline
		Parameter  & Prior & Fiducial value \\
		$\Omega_c$ & $\mathcal{N}_T$ (0.2664, 0.2) & 0.2664 \\
		$\Omega_b$ & $\mathcal{N}$ (0.0492, 0.006) & 0.0492 \\
		$\sigma_8$ & $\mathcal{N}$ (0.831, 0.14) & 0.831 \\
		$h$ & $\mathcal{N}$ (0.6727, 0.063) & 0.6727\\
		$n_s$ & $\mathcal{N}$ (0.9645, 0.08) & 0.9645 \\
		$w_{0}$ &  $\mathcal{N}_T$ (-1.0, 0.9) &  -1.0 \\
		\hline
	\end{tabular}
	\label{tab:prior}
\end{table}
%--------------------------------------------------------------------

\subsection{Noise and survey setting}
We conduct a tomographic study to reproduce the redshift distribution and the expected noise for the LSST-Y10 data release.
Specifically, following \citet{zhang2022transitioning}, we model the underlying redshift distribution using the parametrized Smail distribution \citep{smail1995deep}:
\begin{equation}
    n(z)=\propto z^2 \exp{-(z/z_0)^{\alpha}},
\end{equation}
with $z_0=0.11$ and $\alpha=0.68$ and assuming a photometric redshift error $\sigma_z=0.05(1+z)$ as defined in the LSST DESC Science Requirements Document (SRD, \citet{mandelbaum2018lsst}).
The galaxy sources are divided into 5 tomographic bins, each containing an equal number of galaxies. 
For each redshift bin, we assume Gaussian noise with mean zero and variance given by
 \begin{equation}
     \sigma^2_n= \frac{\sigma_e^2}{A_{pix}n_{gal}},
 \end{equation}
where we set the shape noise $\sigma_e = 0.26$ and the galaxy number density $n_{gal}=27$ arcmin$^{-2}$. Both the shape noise and galaxy number density values are obtained from SRD. The pixel area is given by $A_{pix}\approx$. 
Figure  xx illustrates the resulting source redshift distribution. \autoref{tab:survey_spec} provides a summary of the survey specifications.
\begin{table}
	\centering
	\caption{ LSST Y10 source galaxy specifications in our analysis. All values are based on the LSST DESC SRD.}
	\begin{tabular}{lc} 
             \hline \hline
		Redshift binning & 5 bins \\
		Redshift distribution ($z_{0}, \alpha$) & (0.11, 0.68)  \\
		Number density $n_s$ & 27/arcmin$^2$ \\
		Shape noise $\sigma_e$ & 0.26 \\
		Redshift error $\sigma_z$ &0.05(1+z)  \\
		\hline
	\end{tabular}
	\label{tab:survey_spec}
\end{table}
%--------------------------------------------------------------------
\section{Inference over forward simulation models}
\subsection{Map sampling}


%--------------------------------------------------------------------
\subsection{Explicit Inference}
\subsubsection{Power spectrum}
\subsubsection{Full field and HMC}
%--------------------------------------------------------------------
\subsection{Implicit Inference}

\subsubsection{Benchmark compression scheme}
There are several approaches available for training the compressor, and this section aims to provide an overview of these different methods used in previous works. 
In \autoref{Sec:results}, we will compare the outcomes of these various strategies. Specifically, using the same prior $p(\bm {\theta})$ for the cosmological parameter $\bm{\theta}$ and the simulator described in \autoref{Sec:results}, which samples $\bm {x}$ from $p(\bm{x}| \bm{\theta})$ and produces a set of mock observations $\bm{x_0}$, we will compare the posterior distribution $p(\bm{\theta}|\bm{x})$ obtained form each inference pipeline. This analysis will help to determine the optimal compression strategy, enabling us to extract the maximum amount of information possible given the fixed cosmological parameters.

\paragraph{\textcolor{violet}{Mean Square Error (MSE)}}
One of the most straightforward approaches adopted to train a Neural Network consists of minimizing the $L_2$ norm, or Mean Square Error (MSE). In this section, we demonstrate that this approach is equivalent to training the model to estimate the mean of the posterior distribution, namely:
\begin{equation}\label{Eq:mean_mse}
\left \langle \bm {\theta} \right \rangle_{p(\bm {\theta}|\bm{d})}=\operatorname*{argmin}_{\mathcal{F}(\bm{d})}\mathbb{E}_{p(\bm {\theta}|\bm {d})}[\left\Vert \bm {\theta}-\mathcal{F}(\bm{d})
 \right \Vert_{2}^{2}].
\end{equation}
To demonstrate \autoref{Eq:mean_mse}, we need to minimize the expected value of the $L_2$ norm with respect to $\mathcal{F}(\bm{d})$. Let us consider its derivative:
\begin{align}\label{Eq:moment_1}
   & \frac{\partial}{\partial \mathcal{F}(\bm{d}) }  \mathbb{E}_{p(\bm {\theta}|\bm{d}))}[(\bm {\theta}-\mathcal{F}(\bm{d}))^2] =  \\
    &
    \frac{\partial}{\partial \mathcal{F}(\bm{d}) }  \mathbb{E}_{p(\bm {\theta}|\bm {d})}[\bm {\theta}^2+\mathcal{F}(\bm{d})^2-2\bm {\theta}\mathcal{F}(\bm{d})] = \nonumber \\
    &
    \frac{\partial}{\partial \mathcal{F}(\bm{d}) }  [
    \mathbb{E}_{p(\bm {\theta}|\bm {d})}[\bm {\theta}^2]+\mathcal{F}(\bm{d})^2-2\mathcal{F}(\bm{d})\mathbb{E}_{p(\bm {\theta}|\bm {d})}[\bm {\theta}]]= \nonumber \\
    &2\mathcal{F}(\bm{d})-2 \mathbb{E}_{p(\bm {\theta}|\bm {d})}[\bm {\theta}]. \nonumber
\end{align}
Setting it equal to zero, we obtain the critical value:
\begin{equation}
    \mathcal{F}(\bm{d})= \mathbb{E}_{p(\bm {\theta}|\bm {d})}[\bm {\theta}]. 
\end{equation}
Considering the second-order derivative:
\begin{equation}\label{Eq:minimum_mean}
   \frac{\partial^2}{\partial^2 \mathcal{F}(\bm{d})}  \mathbb{E}_{p(\bm {\theta}|\bm {d})}[(\bm {\theta}-\mathcal{F}(\bm{d}))^2]=2, 
\end{equation}
we can assert that the critical value $\mathcal{F(\bm(\theta))}$ is also a minimum. 
Since 
\begin{equation}
    \mathbb{E}_{p(\bm {\theta}|\bm {d})}[\bm {\theta}|\bm{d}]= \left \langle \bm {\theta} \right \rangle_{p(\bm {\theta}|\bm{d})}
\end{equation}
it follows from \autoref{Eq:minimum_mean} the \autoref{Eq:mean_mse}.
%--------------------------------------------------------------------
\paragraph{\textcolor{violet}{Maximum Absolute Error (MAE)}}
Instead of minimizing the $L_2$ norm, these approaches focus on the $L_1$ norm or Mean Absolute Error (MAE). In this section, we demonstrate that this is equivalent to training the model to estimate the median of the posterior distribution.
By definition, the median of a probability density function $p(x)$ is a real number $m$ that satisfies:
\begin{equation}\label{Eq:definition_median}
\int_{\infty}^{m} p(x)dx=\int_{m}^{\infty}p(x)dx=\frac{1}{2}.
\end{equation}. 
The expectation value of the mean absolute error is defined as:
\begin{equation}
    \mathbb{E}[|x-m|]= \int_{\infty}^{\infty}p(x)|x-m|dx  
\end{equation}
which can be decomposed as
\begin{equation}
        \int_{\infty}^{m}p(x)|x-m|dx +\int_{m}^{\infty}p(x)|x-m|dx .
\end{equation}
To minimize this function with respect to $m$, we need to compute its derivative:
\begin{equation}\label{Eq:absolute_median}
    \frac{d\mathbb{E}[|x-m|]}{dm}=
    \frac{d}{dm}\int_{\infty}^{m}p(x)|x-m|dx +\frac{d}{dm}\int_{m}^{\infty}p(x)|x-m|dx. 
\end{equation}
Considering that $|x-m|=(x-m)$ for $m\le x$ and $|x-m|=(m-x)$ $m\ge x$, 
we can write \autoref{Eq:absolute_median} as:
\begin{equation}
    \frac{d\mathbb{E}[|x-m|]}{dm}=
    \frac{d}{dm}\int_{\infty}^{m}p(x)(m-x)dx +\frac{d}{dm}\int_{m}^{\infty}p(x)(x-m)dx .
\end{equation}
Using the Leibniz integral rule, we get:
\begin{align}
    &\frac{d\mathbb{E}[|x-m|]}{dm}= \\
    &
    p(x)(m-m)\frac{dm}{dm}+\int_{\infty}^{m}\frac{\partial}{\partial m}[p(x)(m-x)]dx + \nonumber \\
    & - p(x)(m-m)\frac{dm}{dm}+\int_{m}^{\infty}\frac{\partial}{\partial m}[p(x)(m-x)]dx \nonumber .
\end{align}
Setting the derivative to zero, we obtain:
\begin{equation}
    \frac{d\mathbb{E}[|x-m|]}{dm}= \int_{\infty}^{m} p(x)dx-\int_{m}^{\infty}p(x)dx =0.
\end{equation}
Thus,
\begin{equation}
\int_{\infty}^{m} p(x)dx=\int_{m}^{\infty}p(x)dx .
\end{equation}
Considering that
\begin{equation}
\int_{\infty}^{m} p(x)dx+\int_{m}^{\infty}p(x)dx=1,
\end{equation}
we obtain 
\begin{equation}\label{Eq:definition_median}
\int_{\infty}^{m} p(x)dx=\int_{m}^{\infty}p(x)dx=\frac{1}{2}.
\end{equation}
%--------------------------------------------------------------------
\paragraph{\textcolor{violet}{Gaussian Negative Log-Likelihood (GNLL)}}
%--------------------------------------------------------------------
\paragraph{\textcolor{violet}{Information Maximising Neural Networks (IMNNs)}}
%--------------------------------------------------------------------
\paragraph{\textcolor{violet}{Variational Mutual Information Maximization (VMIM)}}
The Variational Mutual Information Maximization (VMIM) has been used for cosmological inference problems for the first time by \citet{jeffrey2021likelihood}. This approach aims to maximize the mutual information $I(\bm{t}, \bm {\theta})$ between the cosmological parameters $\bm{\theta}$ and the summary statistics $\bm t$ by training a neural network with parameters $\phi$. 
Mathematically, it is defined as follows:
\begin{align}\label{Eq:mutual_information}
    I(\bm{t}, \bm {\theta}) &= D_{KL}(p(\bm {t}, \bm {\theta})||p(\bm {t})p(\bm {\theta})) \\ \nonumber
    &= \int d^n \bm{\theta} d^n \bm{t} p(\bm t, \bm \theta)\log{\left( \frac{ p(\bm {t}, \bm {\theta})}{ p(\bm {t}) p(\bm {\theta})} \right)} \\ \nonumber
    &= \int d^n \bm{\theta} d^n \bm{t} p(\bm t, \bm {\theta})\log{\left( \frac{ p(\bm {\theta} | \bm {t} )}{ p(\bm {\theta})} \right)} \\ \nonumber
        &= \int d^n \bm{\theta} d^n \bm{t} p(\bm t, \bm {\theta})\log{p(\bm {\theta} | \bm {t} )} - \int d^n \bm{\theta}  d^n \bm{t} p(\bm t, \bm {\theta})\log{p(\bm {\theta})} \\ \nonumber
    &= \int d^n \bm{\theta} d^n \bm{t} p(\bm t, \bm {\theta})\log{p(\bm {\theta} | \bm {t} )} - \int d^n \bm{\theta} p(\bm {\theta})\log{p(\bm {\theta})} \\ \nonumber
    &= \mathbb{E}_{p(\bm {t}, \bm {\theta})} [\log{p(\bm {\theta} | \bm {t} )}]- \mathbb{E}_{p(\bm {\theta})} [\log{p(\bm {\theta})}] \\ \nonumber
    &= \mathbb{E}_{p(\bm {t}, \bm {\theta})} [\log{p(\bm {\theta} | \bm {t} )}]- H(\bm {\theta});
\end{align}

where $D_{KL}$ is the Kullback-Leibler divergence \citep{kullback1951information}, $p(\bm {t}, \bm {\theta})$ is the joint probability distribution of summary statistics and cosmological parameters, and $H(\bm {\theta})$ represents the \textit{entropy} of the distribution of cosmological parameters.  
In other words, the mutual information measures the amount of information contained in the summary statistics $\bm t$ about the cosmological parameters $\bm \theta$.
Given the original high-dimensional data vector $\bm d$, the information $\bm t$ is extracted using $\bm {t}=F_{\bm {\phi}}(\bm {d})$. 
The goal is to find the parameters $\bm {\phi}$ that maximize the mutual information between the summary and cosmological parameters:
\begin{equation}
   \bm {\phi}^*= \operatorname*{argmax}_{\bm {\phi}} I(F_{\bm {\phi}}(\bm{d}), \bm {\theta}).
\end{equation}
However, the mutual information expressed in \autoref{Eq:mutual_information} is not tractable. To overcome this limitation,  several approaches have been developed, relying on tractable bounds that allow the training of a deep neural network to optimize the mutual information. In this work, we adopt the same strategy used by \citet{jeffrey2021likelihood} and use the variational lower bound \citep{barber2003information}:
\begin{equation}\label{Eq:variational_lower_bound}
    I(\bm{t}, \bm{\theta}) \ge \mathbb{E}_{p(\bm {t}, \bm {\theta})} [\log{q(\bm {\theta} |\bm{t} ; \bm{\phi}')}]- H(\bm {\theta}).
\end{equation}
Here, the variational conditional distribution $\log{q(\bm {\theta} |\bm{t} ; \bm{\phi}')}$ is introduced to approximate the true posterior distribution $p(\bm{\theta}|\bm {t})$. 
Since the entropy of the cosmological parameters is constant for a fixed training set, the optimization problem based on the lower bound in  \autoref{Eq:variational_lower_bound} can be written as:
\begin{equation}
    \operatorname*{argmax}_{\bm {\phi}, \bm {\phi}'}\mathbb{E}_{p(\bm {d}, \bm {\theta})} [\log{q(\bm {\theta} |F_{\bm {\phi}}(\bm {d}) ; \bm{\phi}')}].
\end{equation}

\subsubsection{Inference strategy}
%--------------------------------------------------------------------


\begin{center}
\begin{table*}
\begin{tabular}{ |p{3.5cm}|p{2cm}|p{3cm}|p{2.5cm}|p{4cm}|  }
 \hline
Reference & \makecell{Architecture\\ compressor}  & \makecell{Loss function} & \makecell{Inference \\ strategy} & \makecell{Output compressor}   \\
 \hline
            \citet{2018PhRvD..97j3515G} & \makecell{CNN} & \makecell{MAE} &  \makecell{Likelihood \\ analysis} &\makecell{$\sigma_8, \Omega_m$}  \\
 \hline
            \citet{fluri2018cosmological} & \makecell{CNN} & \makecell{MAE} & \makecell{Likelihood \\ analysis} &  \makecell{$\log{(\sigma_{\Omega_m}^2)},\log{(\sigma_{\sigma_8}^2)}$,
             \\
            $\tanh^{-1}{(\text{Corr}(\sigma_8,\Omega_m))}$,
             \\
            $\sigma_8,\Omega_m$} 
\\
\hline     
\rowcolor{lightgray}
            \citet{fluri2019cosmological} & \makecell{CNN} & \makecell{GNLL} & \makecell{Likelihood \\ analysis} & \makecell{$\sigma_8, \Omega_m, A_{IA}/10, L$  \\ with  $L:\Sigma^{-1}=LL^{T}$}
\\
\hline            
            \citet{ribli2018improved} & \makecell{CNN} & \makecell{MSE}  &\makecell{RMSE for \\ evaluation}  & \makecell{$\sigma_8, \Omega_m$}   
\\
\hline            
            \citet{ribli2019weak} & \makecell{CNN} & \makecell{MAE} & \makecell{Likelihood \\ analysis} & \makecell{$\sigma_8, \Omega_m$}   
\\            
\hline             
            \citet{PhysRevD.102.123506} & \makecell{CNN} & \makecell{MAE} & \makecell{Likelihood \\ analysis} & \makecell{$\sigma_8, \Omega_m$}   
\\
\hline 
\rowcolor{lightgray}
            \citet{jeffrey2021likelihood} & \makecell{\makecell{CNN} \\\makecell{CNN}+NF} & 
            \makecell{MSE \\ VMIM}
            & \makecell{PyDelfi} & \makecell{$\phi: F_{\phi}(\bm{d})=\bm{\theta}$ \\ with $\bm{\theta}=\Omega_m, \sigma_8$}    
\\            
\hline             
            \citet{fluri2021cosmological} & \makecell{GCNN} & \makecell{IMNN} & \makecell{GPABC} &  
\\
\hline      
\rowcolor{lightgray}

            \citet{fluri2022full} & \makecell{GCNN} & \makecell{IMNN} & \makecell{GPABC} &  
\\            
\hline 
            \citet{lu2022simultaneously} & \makecell{CNN} & \makecell{MSE} & \makecell{Likelihood \\ analysis}  & \makecell{$\Omega_m,S_8, A_{IA}/10,$ \\ rescaled \\ baryonic parameters \\ $(M_c,M_{1,0}, \eta, \beta)$}   
\\           
\hline 
            \citet{kacprzak2022deeplss} & \makecell{CNN} & \makecell{GNLL}  & \makecell{Likelihood \\ analysis}  & \makecell{WL: $\Sigma, \Omega_m, \sigma_8, A_{IA}, \eta_{IA}$ \\  GC:$\Sigma, \Omega_m, \sigma_8, b_g, r_g,\eta_{b_{g}}$ }
          
\\
\hline 
\rowcolor{lightgray}
            \citet{lu2023cosmological} & \makecell{CNN} & \makecell{MSE} & \makecell{Likelihood \\ analysis}  & \makecell{$\Omega_m,S_8,A_{IA}/10,$\\ rescaled baryonic \\ parameters: $M_c,M_{1,0}, \eta, \beta$} 
\end{tabular}
\label{tab:biblio_survey}
\end{table*}
\end{center}



\citep{Campagne_2023}
%--------------------------------------------------------------------
\section{Results}\label{Sec:results}
%--------------------------------------------------------------------
\section{Conclusions}
%--------------------------------------------------------------------
%--------------------------------------------------------------------
\begin{acknowledgements}
This work was granted access to the HPC/AI resources of IDRIS under the allocation 2022-AD011013922 made by GENCI.
\end{acknowledgements}

\bibliographystyle{aa} % style aa.bst
\bibliography{paper/biblio} 
\end{document}